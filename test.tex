\documentclass[12pt, a4paper, twoside]{article}
\usepackage[T1,T2A]{fontenc}
\usepackage[utf8]{inputenc}
\usepackage[english,bulgarian,ukrainian,russian]{babel}
\usepackage{graphicx}
\usepackage{array}
\usepackage{tabularx}
\usepackage{listings}
\usepackage{color}
\usepackage{minted}
\usepackage{float}
\usepackage{enumitem}
\setlist{nolistsep, itemsep=0.3cm,parsep=0pt}
\usepackage{geometry}
\geometry{top=20mm}
\geometry{bottom=20mm}
\usepackage{csquotes}
\usepackage{soul}



\title{Взаимодействие Windows API\\
с помощью PowerShell}

\author{Боярова Елизавета}
\date{Январь 2024}

\begin{document}

\maketitle

\newpage
\begin{center}
\tableofcontents 
\end{center}

\newpage
\begin{center}
\section{ОБЩИЕ СВЕДЕНИЯ}
\subsection{Microsoft Windows API}
\end{center}

\textbf{Microsoft Windows API} — интерфейс системного программирования для операционных систем Windows. 
Необходим для того, чтобы написанные приложения могли взаимодействовать с операционной системой.\\

\begin{center}
\includegraphics[width=0.8\textwidth]{Images/1.jpg}
\end{center}\\

Windows API (Win32) ориентирован главным образом на язык программирования C, поскольку его предоставляемые функции и структуры данных описаны на этом языке в последних версиях его документации. Однако API может использоваться любым компилятором или ассемблером языка программирования, способным обрабатывать (четко определенные) низкоуровневые структуры данных вместе с предписанными соглашениями о вызовах для вызовов и обратных вызовов.

\begin{center}
\subsection{PowerShell}
\end{center}

\textbf{PowerShell} – это стандартизированная оболочка командной строки, которая открывает доступ к более гибкому управлению компьютером и его настройкам. Это та же командная строка, но с большими возможностями для конфигурирования операционных систем семейства MS Windows. Проще говоря, это своего рода универсальный инструмент администрирования. 
В отличие от большинства оболочек, которые только принимают и возвращают текст, PowerShell принимает и возвращает объекты \definecolor{light-gray}{rgb}{0.8,0.8,0.8}
\colorbox{light-gray}{.NET}.\\

\begin{flushleft}Это решение предлагает следующие возможности:\end{flushleft}
\begin{itemize}
     \item надежный журнал командной строки;
     \item заполнение нажатием клавиши TAB и подстановка команд;
     \item поддержка псевдонимов команд и параметров;
     \item создание конвейера для объединения команд;
     \item система справки консоли, похожая на страницы man в Unix.
\end{itemize}

\begin{center}
\subsection{Команды PowerShell}
\end{center}
\enlargethispage{3\baselineskip}

Оболочка PowerShell имеет собственный набор команд, именуемый командлетами (от английского «cmdlet»). Они формируются с помощью шаблона «Глагол-Существительное», или «Действие-Объект». 

Например, \textit{Get-Services} и \textit{Start-Process}.\\

\begin{flushleft}
\textbf{Основные команды языка Windows PowerShell}
\end{flushleft}

Ниже таблице перечислены основные команды PowerShell и их аналоги в *nix-подобных системах и CMD.EXE.

\begin{table}[H]
    \centering
    \begin{tabular}{ | m{3cm} | m{3cm}| m{3cm} | m{3cm} | } 
    \hline
    Командлет (псевдоним) & Команда в \*nix & Команда в CMD.exe & Описание\\
    \hline
    Get-Location (pwd) & pwd &  & Выводит путь до текущего каталога\\
    \hline
    Set-Location (cd) & cd & cd & Меняет текущий каталог \\
    \hline
    Get-ChildItem (ls) & ls & dir & Выводит содержимое текущего каталога\\ 
    \hline
    Get-ChildItem & find & find & Производит поиск файлов по заданным критериям\\ 
    \hline
    Copy-Item (cp)  & cp & cp & Копирует файл\\ 
    \hline
    Remove-Item (rm) & rm & rm & Удаляет файл\\ 
    \hline
    \end{tabular}
\end{table}

\begin{center}
\subsection{Синтаксис}
\end{center}

После имени самого командлета следует указание параметров и их значений. Между всеми частями команды следует проставлять пробелы.

Пример: \textit{Set-Location -LiteralPath C:\ -PassThru}\\

Разберем пример:\\

\textit{Set-Location} - буквально «вызвать команду». Позволяет выполнить указанный блок сценария.\\

\textit{-LiteralPath C:\ } – здесь передаем блок сценария, в котором используется команда Set-Location для перехода в каталог C:\.\\

\textit{-PassThru} – по умолчанию командлет Invoke-Command не возвращает результат выполнения. Этот параметр указывает на необходимость вывода информации о местоположении, в которое был выполнен переход с помощью команды Set-Location.\\

\newpage
Регистр букв в командах PowerShell не имеет значения, команды будут выполнятся в любом регистре. Когда в одной строке объединены несколько команд, они разделяются точкой с запятой \definecolor{light-gray}{rgb}{0.8,0.8,0.8}
\colorbox{light-gray}{;} .\\

Если команда вышла слишком длинной, то для разделения на несколько строк используйте символ обратного апострофа \definecolor{light-gray}{rgb}{0.8,0.8,0.8}
\colorbox{light-gray}{`}, вместо переноса. Новую строку можно создать, нажав \textbf{Shift + Enter} (для переноса строки ниже текущей) или \textbf{Ctrl + Enter} (для переноса строки выше текущей). \\

\newpage
\begin{center}
\section{НАЧАЛО РАБОТЫ}
\subsection{Как запустить PowerShell в Windows}
\end{center}

По умолчанию PowerShell уже установлен на вашем компьютере. 
Если нет, то воспользуйтесь инструкциями для установки, предоставленными Microsoft.   

Существуют два основных способа запуска PowerShell в системе Windows: 
через меню «Пуск» и с помощью приложения «Выполнить».\\

\begin{flushleft}
    \textbf{Через меню «Пуск»}:
\end{flushleft}\\

\begin{center}
\includegraphics[width=0.8\textwidth]{Images/2.png}
\end{center}\\

\newpage
\begin{flushleft}
    \textbf{Через приложение «Выполнить»}:
\end{flushleft}

Используйте комбинацию клавиш \textbf{Win + R}. В появившемся окне введите powershell и нажмите кнопку \textbf{ОК}.\\

\begin{center}
\includegraphics[width=0.8\textwidth]{Images/3.png}
\end{center}\\

\begin{center}
\subsection{Взаимодействие с функциями Windows API с помощью PowerShell}
\end{center}

В Windows PowerShell существует три способа взаимодействия с функциями Windows API:\\
\begin{itemize}
    \item С помощью командлета \(Add-Type\) для компиляции кода C\#;
    \item С помощью ссылки на приватный тип в .NET Framework вызвать данный метод;
    \item С помощью рефлексии, чтобы определить метод, вызывающий функцию Windows API.
\end{itemize}

\begin{center}
\subsection{Использование командлета Add-Type для вызова функции}
\end{center} 

Ключ к взаимодействию с функциями WinAPI через PowerShell – знать, как преобразовывать типы в С/С++ функциях в их эквиваленты в фреймворке .NET. Для этой цели используется сайт \( pinvoke.net.\) \\

На сайте \( pinvoke.net \) в поисковой строке нужно указать имя искомой функции из WinAPI, например, \( CopyFile \). Результатом поиска станет сигнатура эквивалентной C\# функции:\\

\begin{center}
\textit{[DllImport(“kernel32.dll”, CharSet = CharSet.Unicode)] \\
static extern bool CopyFile(string lpExistingFileName, \\
string lpNewFileName, bool bFailIfExists); }
\end{center}

Далее необходимо использовать командлет \(Add-Type\) для взаимодействия с найденной C\# функцией. Этот командлет позволяет на лету скомпилировать добавленный в него C\# код.\\
\newpage
Добавим найденную функцию \(CopyFile\) в PowerShell скрипт:\\

\begin{center}
\textit{\$MethodDefinition = @’}


\textit{[DllImport(“kernel32.dll”, CharSet = CharSet.Unicode)]}


\textit{public static extern bool CopyFile(string lpExistingFileName, string lpNewFileName, bool bFailIfExists);}


\textit{‘@}


\textit{Add-Type -MemberDefinition \$MethodDefinition -Name ‘Kernel32’ -Namespace ‘Win32’ -PassThru}\\
\end{center}


Переменная \(MethodDefinition\) содержит сигнатуру функции \(CopyFile\), найденную на сайте \(pinvoke.net\) с одним изменением: функция \(CopyFile\) объявлена как \(public\), так как функции, который добавляются с помощью \(Add-Type\) должны быть публичными для облегчения взаимодействия с PowerShell.\\

Далее, с помощью вызова Add-Type обеспечивается доступ к C\# коду через указание имени типа и пространства имен. После вызова Add-Type вы можете использовать новый тип в PowerShell как [Win32.Kernel32].\\


Например, теперь можно вызвать функцию \(CopyFile\) из PowerShell с помощью команды:
\begin{center}
\textit{[Win32.Kernel32]::CopyFile("C:\Documents\file\_1.txt ", "C:\Documents\file\_2.txt", \$False)\\}
\end{center}
Эта команда скопирует файл \textbf{file\_1.txt} в файл \textbf{file\_2.txt}.\\


Таким образом мы смогли вызвать функцию CopyFile из WinApi в PowerShell.








\end{document}